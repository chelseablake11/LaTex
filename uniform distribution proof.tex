\documentclass{article}
\title{Uniform Distribution Proof}
\author{Chelsea Blake}



\begin{document}
\maketitle

\begin{proof}
\begin{equation}
\textrm{Show that if} $\mu_n=\frac{1}{n} \sum \limits_{k=1}^{n} \delta_{\frac{k}{n}}$ 
\begin{center}
\textrm{then} 

$\lim \limits_{n \rightarrow \infty} \mu_n [a,b] = b-a$
\end{center}
\end{equation}
%---- fix this ------------------------------%
\begin{center}
Where $\delta_{\frac{k}{n} ([0,a])=$
\\
\begin{cases}  
      1 & \textrm{ for $\frac{k}{n} \in [0,a]$}  \\
      0 & \textrm{ for $\frac{k}{n} \not \in [0,a]$} 
\end{cases}
\end{center}
%--------------------------------------------------%
\\
\\
\textit{First lets look at an example.}
\\
\begin{center}
Case 1: $[0,1]$
\\
$\mu_n [0,1]=\frac{1}{n} \sum \limits_{k=1}^n \delta_{\frac{k}{n}}([0,1])$ 
\\
We see that $ \delta_{\frac{k}{n}}([0,1])=1 \; \; \forall \; k=1,2,3,$...
\\

$\frac{1}{n} \sum 1 = \frac{1}{n} n = 1= 1-0 = b-a$ 
So Case 1 works.
\end{center}
\\
\\
\textit{Now we try to make a general proof.} 
\\
\\
Given any $n$ and $a$, there is a $l$ such that
\\
$a-\frac{1}{n} \le \frac{l}{n} \le a \le \frac{l+1}{n} \le a + \frac{1}{n}$
\\
\\
\textit{Firstly,} we will look at
\\
\begin{equation}
$\mu_n [0,a]= \frac{1}{n} \sum \limits_{k=1}^{n} \delta_{\frac{k}{n}} [0,a]$ 
\end{equation}
\\
We can then break the sum into to parts, $\sum \limits_{k=1}^{n}= \sum \limits_{k=1}^{l} + \sum \limits_{k=l+1}^{n}$ 
\\
$\frac{1}{n} (\sum \limits_{k=1}^{l} \delta_{\frac{k}{n}}[0,a] + \sum \limits_{k=l+1}^{n} \delta_{\frac{k}{n}} [0,a])$
\\
Where $\sum \limits_{k=1}^{l} \delta_{\frac{k}{n}}[0,a]=\sum \limits_{k=1}^{l} 1= l$ because $\frac{l}{n} \le a$ 
\\
and $\sum \limits_{k=l+1}^{n} \delta_{\frac{k}{n}} [0,a]=0$ because $\frac{l+1}{n}}} > a$ 
\\
We then see that equation (2) equals
\\
$\frac{1}{n}(l+0)=\frac{1}{n} (l)=\frac{l}{n}$
\\
\\
We now want to take $\lim \limits_{n \rightarrow \infty}$ of $\frac{l}{n}$
\\
\\
Looking back at the inequality stated above we see that given any $n$ and $a$, there is a $l$ such that
\\
$a-\frac{1}{n} \le \frac{l}{n} \le a \le \frac{l+1}{n} \le a + \frac{1}{n}$
\\
\\
Taking the limit of each side of the inequality we see that
\\
$\lim \limits_{n \rightarrow \infty} \; a-\frac{1}{n} \rightarrow a$
\\
and  $\lim \limits_{n \rightarrow \infty} \; a+\frac{1}{n} \rightarrow a$
\\
\\
Then using the Squeeze Theorem:
\\
$\lim \limits_{n \rightarrow \infty}$ of $a-\frac{1}{n} \le \frac{l}{n} \le a + \frac{1}{n}= \lim \limits_{n \rightarrow \infty} \; a-\frac{1}{n} \rightarrow a \le \lim \limits_{n \rightarrow \infty} \frac{l}{n} \le \lim \limits_{n \rightarrow \infty} \; a+\frac{1}{n} \rightarrow a = a$
\\
\\
We can finally conclude that  $\lim \limits_{n \rightarrow \infty}$ of $\frac{l}{n}=a$ which proves equation 2 to be true.
\\
if $\mu_n [0,a]= \frac{1}{n} \sum \limits_{k=1}^{n} \delta_{\frac{k}{n}} [0,a]$ then $\lim \limits_{n \rightarrow \infty} \mu_n (0,a) = a-0$
\\
\\
\\
We can similarly show that this proof is true for $(0,a),(0,a]$ and $[0,a)$ The change in open and closed set will only change whether the inequalities are strict or not in the inequality that related $n, l,$ and $a$ (Shown below). 
\\
\\
Given any $n$, there is a $l$ such that We know that for $[0,a]$,
\\

$a-\frac{1}{n} \le \frac{l}{n} \le a \le \frac{l+1}{n} \le a + \frac{1}{n}$

\\
\\
However we see that that, for example, Given any $n$, there is a $l$ such that We know that for $(0,a)$,
\\
$a-\frac{1}{n} < \frac{l}{n} < a < \frac{l+1}{n} < a + \frac{1}{n}$
\\
\\
Putting this inequality into the proof above, we will still get the same answer that if $\mu_n [0,a]= \frac{1}{n} \sum \limits_{k=1}^{n} \delta_{\frac{k}{n}} [0,a]$ then $\lim \limits_{n \rightarrow \infty} \mu_n (0,a) = a-0$
\\
This will be true for all arrangements of the open/closed set.
\\
\\
\\
\begin{equation}
\textrm{Because $a$ is a variable, we can show the same proof for $[0,b]$ stating if}
$\mu_n [0,b]= \frac{1}{n} \sum \limits_{k=1}^{n} \delta_{\frac{k}{n}} [0,b]$ then $\lim \limits_{n \rightarrow \infty} \mu_n [0,b] = b-0$
\end{equation}
\\
\\
\\
%Finally, we proved above that $\mu [a,b]= \mu [0,b]- \mu [0,a]$ and therefor,
%\\
%if $\mu_n=\frac{1}{n} \sum \limits_{k=1}^{n} \delta_{\frac{k}{n}}$ then $\lim %\limits_{n \rightarrow \infty} \mu_n [a,b] = (b-0)-(a-0)=b-a$ 
%\\
%\\

We will now show that $\mu [a,b]= \mu [0,b]- \mu [0,a]$

\\
\begin{equation}

$[0,b]=[0,a] \cup (a,b]$

\end{equation}
\\
\\
\\
Using equation 4, we can infer that $\mu_n[0,b]= \mu_n[0,a] \cup \mu_n(a,b]$
\\
\\
We then manipulate the equation and take $\lim \limits_{n \rightarrow \infty}$ to get
\\
$\lim \limits_{n \rightarrow \infty} \mu_n [0,b] - \lim \limits_{n \rightarrow \infty} \mu_n[0,a] = \lim \limits_{n \rightarrow \infty} \mu_n (a,b]$ We know from above that $\lim \limits_{n \rightarrow \infty} \mu_n [0,b]=b$ and $\lim \limits_{n \rightarrow \infty} \mu_n[0,a] = a$ 
\\
\\
\\
so therefore, $\lim \limits_{n \rightarrow \infty} \mu_n (a,b]=b-a$
\\
\\
And similarly equation 1. 


\end{proof}




\end{document}
