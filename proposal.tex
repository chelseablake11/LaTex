\documentclass[ebook,12pt,oneside,openany]{memoir}
\usepackage[utf8x]{inputenc}
\usepackage[english]{babel}
\usepackage{mathtools}
\usepackage{url}

% for placeholder text
\usepackage{lipsum}

\title{Pythagorean Fifth and Measure Theory- Proposal}
\author{Chelsea Blake}

\begin{document}
\maketitle

Music, at it's most basic form, is a group of frequencies. Some frequencies sound better put together than other frequencies, this is called harmonizing. The ratio of frequencies for good sounding harmonies, as believed for many, would be a ratio of low, simple numbers, like a $3:2$ ratio. The simple $3:2$ ratio is known as the Pythagorean fifth and is where my research will start.

In most even tempered turned musical instruments today these simple ratios are not used. Instead irrational forms of $2^{\frac{1}{12}} \; Hz$ is used. For example the frequency of a key in a piano gets multiplied by $2^{\frac{1}{12}}$ of the frequency found on the key before it. The equivalent to the Pythagorean fifth, or a frequency ratio of $3 : 2$, would be to go up seven keys on a keyboard. The change in frequency would then be $2^{\frac{7}{12}}=1.498307 \approx 1.5= \frac{3}{2}$. We see that the irrational multiplier is extremely close to the rational multiplier.

Most instruments in Western music use the even tempered tuning because of the way it works with the musical chords. These chords go C-D-E-F-G-A-B-C and then starts back at C and recycles again. Going up by a degree of $2^{\frac{1}{12}} \; Hz$ is the only way that you hit each chord and then get back to C exactly.  Going from a C chord back to a C chord is called an octave, and has a raise in frequency of $2 \; Hz$. In this research we will be looking at rises by $2^{\frac{7}{12}} \; Hz$ because it is closely related to the ratio $3:2$. Rising by the multiplier $2^{\frac{7}{12}} \; Hz$ is called the circle of fifths. To visualize the circle of fifths we say:
\begin{center}
$(2^{\frac{7}{12}})^n=2^m$ for some $n,m$ (specifically $n=12, m=1$).

\end{center}


Using the rational multiplier of $\frac{3}{2}$  and starting at chord C when the frequency got back to chord C it would be slightly off, making it not chord C at all. A new chord would be created and the frequency would never exactly hit the C chord. in other words 
\begin{center}
$(\frac{3}{2})^n \ne 2^m$ for any $n,m$. 
\end{center}

My research will be on the new chords that are made from the rational multiplier. These chords will be mapped to the unit circle for simplicity.The chords will be points $r$, the frequency will be $f(r)$, and the angle between points on the unit circle will be $\theta$. I will find these points by hand as well as write up a program on Matlab to find these points. I will identify ratios between the $\theta$'s. I will look for the limit as these frequencies get multiplied higher and higher to infinity. I will also prove that even as the frequency goes to infinity, it will never again reach chord C. 

Measure theory is a branch of mathematics that looks at collection, or sets, and puts a positive real number measure to it (Gupta, 1). The sets that will be used for my research will be the frequency points found above and the subset $[0,2 \pi]$. After mapping the frequency points to the number line, I will be looking at the discrete measures to see if it weakly converges to a function. I will also be finding the Lebesgue Measure $\mu_{\mathhbb{L}}$ for the set and using it to find the density $\mu_{\rho}$ at the interval $[a,b]$
Some important definitions to know for my research are
\\

\begin{defintion}

\textbf{Definition 1.} \underline{Absolutely Continuous Measures}
$$\mu(A)= \int_A f(x) \; dx$$ For some $f$.

(Arciero's notes, 1)


\end{definition}


\begin{defintion}

\textbf{Definition 2.} \underline{Discrete Measures}
$$\mu = \sum a_i \delta_{x_i}$$ $$\mu(A)= \sum a_1 \; \; \{i|x_i \in A\}$$


\end{definition}



\begin{definition}


\textbf{Definition 3.} \underline{Weak Convergence:} A sequence of positive measures $\mu_n$ will be said to converge weakly to a measure $\mu_{\infty} \; \; (\mu_n \xrightarrow[ ]{w} \mu_{\infty} $ if $\int \limits_0^{2 \pi} f \; d \mu_n \rightarrow \int \limits_0^{2 \pi} f \; d \mu_{\infty} $ for all continuous functions $f$ with $f(0)=f(2\pi)$.

(Pakula, 0.3).


\end{definition}

\\

\begin{defintion}

\textbf{Definition 4.} \underline{Lebesgue Measures} are obtained from an increasing, right-continuous function $F : \mathbb{R} \rightarrow \mathbb{R}$, and assign to an open interval $[a,b]$ the measure
\\

$$\mu_{\mathhbb{L}} ([a,b])= f(b)-f(a)$$

(Hunter, 30).

\end{definition}
\\
\\



Although just a few examples of Measure Theory, I will be using quite a bit of measure theory to help further my research. My end result will be to find the asymptotic distribution of the frequency points on the unit circle, or on a number line.




\huge{\textbf{References}}


\hangindent \small



Gupta, Maya R. "A Measure Theory Tutorial (Measure Theory for Dummies)." University of Washington, Dept. of EE (n.d.): n. pag. Web. May 2006. <https://www.ee.washington.edu/techsite/papers/documents/UWEETR-2006-0008.pdf>.

Hunter, John K. "Measure Theory." , University of California at Davis (n.d.): n. pag. Web. 2011. <https://www.math.ucdavis.edu/~hunter/measure_theory/measure_notes.pdf>.

Pakula, Lewis. "Measures- an informal, non-rigorous introduction."Notes.28 Jan. 2016.

Tao, Terrence. "An Introduction to Measure Theory." Department of Mathematics, UCLA (n.d.): n. pag. Web. 28 Jan. 2016. <http://www.faculty.jacobs-university.de/poswald/teaching/RealAnal2/handouts/TaoMeasureTheory.pdf>.




\end{document}